\documentclass[lnbip]{svmultln}
\usepackage{makeidx}
\usepackage{tcolorbox}
\usepackage{multirow}
\usepackage{amsmath}
\makeindex

\begin{document}
\mainmatter
\title{Probabilidade Bayesiana}

\author{Flávio Luiz Seixas\inst{1}}
\institute{Instituto de Computação\\
\email{fseixas@ic.uff.br},
\texttt{http://www.ic.uff.br/~fseixas}}

\maketitle

\section{Paradigmas Frequentista e Bayesiano}

O paradigma frequentista admite a probabilidade num contexto restrito a fenômenos que podem ser medidos por frequências relativas. O paradigma Bayesiano entende-se que a probabilidade é uma medida racional e condicional de incerteza. Uma medida do grau de plausibilidade de proposições quaisquer, as quais não precisam necessariamente estar associadas a fenômenos medidos por frequência relativa. Por exemplo, no paradigma Bayesiano admite-se falar da probabilidade de extinção de uma espécie, o que não seria admissível sob o paradigma frequentista.

A inferência estatística é o processo formal utilizado para fazer afirmações genéricas com base em informações parciais. Essas afirmações sáo probabilísticas pois se caracterizam por incluir componentes de incerteza.

Na perspectiva bayesiana, a inferência estatística sobre qualquer quantidade de interesse é descrita como a modificação que se processa nas incertezas à luz de novas evidências.

A conceituação frequentista admite falar em probabilidades somente no contexto de frequências relativas. Em contraste, na conceituação bayesiana, probabilidades quantificam as plausibilidades de proposições ou eventos. Ao atribuir plausibilidades diferenciadas a proposições, a formalização bayesiana de probabilidade estende a lógica dedutiva, restrita a classificar proposições em verdadeiras (probabilidade igual a 1) ou falsas (probabilidade igual a zero), para um conjunto de possibilidades entre estes dois extremos.

O rápido crescimento do uso do paradigma bayesiano em ciências aplicadas ao longo das últimas décadas foi facilitado pelo surgimento de vários programas para efetuar as computações estatísticas. Entre esses, destaca-se o R (programa de livre distribuições e de código aberto).

\section{As Regras de Probabilidade}

A probabilidade será um número real e função de dois argumentos: o evento incerto $E$ e a premissa $H$. Utilizaremos o símbolo $Pr(E|H)$ lido como probabilidade de $E$ dado que $H$ é fato, ou a probabilidade de $E$ condicionada ao fato $H$.

\begin{tcolorbox}[colback=blue!5,colframe=blue!75!black,title=A Lei da convexidade]
    A probabilidade de um evento qualquer $E$, condicionado a $H$ é um número real no intervalo $[0,1]$
    \begin{equation}
        0<Pr(E|H)<1
    \end{equation}
\end{tcolorbox}

\begin{tcolorbox}[colback=blue!5,colframe=blue!75!black,title=A Lei da adição]
    Se $E_1$ e $E_2$ são eventos exclusivos sob $H$, então a probabilidade da união lógica de $E_1 + E_2$ é igual a soma aritmética das suas probabilidades individuais condicionadas a $H$.
    \begin{equation}
        Pr(E_1 + E_2 | H) = Pr(E_1|H) + Pr(E_2|H)
    \end{equation}
\end{tcolorbox}

\begin{tcolorbox}[colback=blue!5,colframe=blue!75!black,title=A Lei do produto]
    Se $E_1$ e $E_2$ são eventos quaisquer então a probabilidade do produto lógico $E_1 E_2$ condicionado a $H$ é o produto da probabilidade de $E_1$ condicionado a $H$ multiplicado pela probabilidade de $E_2$ condicionado a $E_1 H$
    \begin{equation}
        Pr(E_1 E_2 | H) = Pr(E_1|H) \bullet Pr(E_2|E_1 H)
    \end{equation}
\end{tcolorbox}

Nos casos em que estamos tratando de eventos independentes, a lei do produto pode ser reescrita como:
\begin{equation}
    Pr(E_1 E_2 | H) = Pr(E_1|H) \bullet Pr(E_2|H)
\end{equation}

\section{O Teorema de Bayes}

Mutas propriedades do cálculo de probabilidades podem ser deduzidas a partir das três leis básicas indicadas na seção anterior. Depois teoremas adicionais merecem especial destaque, o Teorema da Probabilidade Total e o Teorema de Bayes.

\begin{tcolorbox}[colback=blue!5,colframe=blue!75!black,title=Teorema da Probabilidade Total]
    Seja ${E_1; j=1, ..., m}$ um conjunto de $m$ eventos exclusivos e exaustivos sob $H$, e seja $A$ outro evento qualquer. Então $Pr(A|H)$ pode ser reescrito estendendo a conversa para a inclusão dos $E_j$.
    \begin{equation}
        Pr(A|H) = \sum_{j=1}^{m}{Pr(A|E_j H) \bullet Pr(E_j|H)}
    \end{equation}
\end{tcolorbox}

\begin{tcolorbox}[colback=blue!5,colframe=blue!75!black,title=Teorema de Bayes]
    Sejam $E$ e $F$ dois eventos quaisquer e $Pr(E|H)>0$, então:
    \begin{equation}
        Pr(F|E H) = \frac{Pr(E|F H) \bullet Pr(F|H)}{Pr(E|H)}
    \end{equation}
\end{tcolorbox}

\subsection{Exemplo}

Um estudo de uma mamografia no diagnóstico de câncer é apresentado na Tabela~\ref{tab:tab1}. Os dados foram obtidos experimentalmente sobre a efetividade do exame na detecção de um tumor de mama maligno ou benigno. Por exemplo, se um tumor é maligno $Ca$, a probabilidade de que o exame resulte positivo é $Pr(Pos | Ca) = 0,792$, ou seja, 79,2\%. De forma similar temos $Pr(Neg | Ca') = 0,904$ como a probabilidade de que o exame resulte negativo se o tumor não é maligno $(Ca')$. Os percentuais para faltos positivos e falsos negativos são 9,6\% e 20,8\%, respectivamente.


\begin{table}[h]
    \caption{Resultados dos testes de câncer de mamas}
    \label{tab:tab1}
    \center
    \begin{tabular}{ |c|c|c| } 
    \hline
    \multirow{2}{4em}{Resultado do teste} & \multicolumn{2}{|c|}{Realidade} \\
    & $Ca$ (Tumor maligno) & $Ca'$ (Tumor benigno) \\ 
    \hline
    $Pos$ (Positivo) & 0,792 & 0,096 \\ 
    $Neg$ (Negativo) & 0,208 & 0,904 \\ 
    \hline
    \end{tabular}
\end{table}

Com essa tabela, fez-se a seguinte pergunta: "Suponha que uma paciente pertença a uma população (mesmo grupo etário, hábito alimentar, etc.) na qual a incidência geral de câncer de mama é de 1\%. Detectado um nódulo no seio desta paciente, pede-se uma mamografia para avaliar a possibilidade de que se trate de um tumor maligno; o resultado é positivo. De posse deste conjunto de informações, qual é, em sua opinião, a probabilidade de tratar-se de um tumor maligno?"

Pelo Teorema de Bayes:
\begin{equation}
\begin{split}
Pr(Ca|Pos) & = \frac{Pr(Pos|Ca) \bullet Pr(Ca)}{Pr(Pos)} \\
& = \frac{Pr(Pos|Ca) \bullet Pr(Ca)}{P(Pos|Ca) \bullet P(Ca) + Pr(Pos|Ca') \bullet Pr(P(Ca')} \\
& = \frac{0,792 \bullet 0,01}{0,792 \bullet 0,01 + 0,096 \bullet 0,99} = 0,077
\end{split}
\end{equation}

\end{document}