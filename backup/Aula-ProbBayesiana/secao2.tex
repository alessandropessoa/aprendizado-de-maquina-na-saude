\section{As Regras de Probabilidade}

A probabilidade será um número real e função de dois argumentos: o evento incerto $E$ e a premissa $H$. Utilizaremos o símbolo $Pr(E|H)$ lido como probabilidade de $E$ dado que $H$ é fato, ou a probabilidade de $E$ condicionada ao fato $H$.

\begin{tcolorbox}[colback=blue!5,colframe=blue!75!black,title=A Lei da convexidade]
    A probabilidade de um evento qualquer $E$, condicionado a $H$ é um número real no intervalo $[0,1]$
    \begin{equation}
        0<Pr(E|H)<1
    \end{equation}
\end{tcolorbox}

\begin{tcolorbox}[colback=blue!5,colframe=blue!75!black,title=A Lei da adição]
    Se $E_1$ e $E_2$ são eventos exclusivos sob $H$, então a probabilidade da união lógica de $E_1 + E_2$ é igual a soma aritmética das suas probabilidades individuais condicionadas a $H$.
    \begin{equation}
        Pr(E_1 + E_2 | H) = Pr(E_1|H) + Pr(E_2|H)
    \end{equation}
\end{tcolorbox}

\begin{tcolorbox}[colback=blue!5,colframe=blue!75!black,title=A Lei do produto]
    Se $E_1$ e $E_2$ são eventos quaisquer então a probabilidade do produto lógico $E_1 E_2$ condicionado a $H$ é o produto da probabilidade de $E_1$ condicionado a $H$ multiplicado pela probabilidade de $E_2$ condicionado a $E_1 H$
    \begin{equation}
        Pr(E_1 E_2 | H) = Pr(E_1|H) \cdot Pr(E_2|E_1 H)
    \end{equation}
\end{tcolorbox}

Nos casos em que estamos tratando de eventos independentes, a lei do produto pode ser reescrita como:
\begin{equation}
    Pr(E_1 E_2 | H) = Pr(E_1|H) \cdot Pr(E_2|H)
\end{equation}
